
\documentclass{article}

\setlength\textheight{9.1in}
\setlength\textwidth{6.5in}
\setlength\headheight{0pt}
\setlength\headsep{0pt}
\setlength\topmargin{0.0in}
\evensidemargin .05in
\oddsidemargin  .05in

\begin{document}

\begin{center}
{\Large\bf List of Publications \vspace{4mm} \\ Ofer Arieli \vspace{4mm} \\ }
\end{center}


\section*{Books}

\begin{enumerate}

   \item A.Avron, O.Arieli, A.Zamansky. {\em Theory of Effective Propositional Paraconsistent Logics\/}.
          Studies in Logic, volume~75 (sub-series: Mathematical Logic and Foundations), College Publications, 2018
         (ISBN:~978-1-84890-270-1).

\end{enumerate}


\section*{Articles in Scientific Journals}

\begin{enumerate}

    \item O.Arieli, A.Avron.
          Reasoning with logical bilattices. {\em Journal of Logic, Language, and Information\/}~5(1),
          pages~25--63, 1996.

    \item O.Arieli, A.Avron.
          The value of the four values. {\em Artificial Intelligence\/}
          102(1), pages 97--141, 1998.

    \item O.Arieli, A.Avron.
          A model theoretic approach to recover consistent data from
          inconsistent knowledge-bases. {\em Journal of Automated Reasoning\/}~22(3),
          pages~263--309, 1999.

    \item O.Arieli.
          Four-valued logics for reasoning with uncertainty in
          prioritized Data. In: {\em Information, Uncertainty, Fusion.\/}
          B.Bouchon-Meunier, R.R.Yager, and L.A.Zadeh, editors, Kluwer
          Academic Publishers, pages~293--304, 1999.

    \item O.Arieli, A.Avron.
          Bilattices and paraconsistency. In: {\em Frontiers of
          Paraconsistent Logic.\/} D. Batens, C. Mortensen, G. Priest, and
          J. Van Bendegem, editors, Studies in Logic and Computation~8,
          Research Studies Press, pages~11--27, 2000.

    \item O.Arieli, A.Avron.
          General patterns for nonmonotonic reasoning: from basic
          entailments to plausible relations. {\em The Logic Journal of
          the Interest Group in Pure and Applied Logics\/} (IGPL)~8(2),
          pages~119--148, 2000.

    \item O.Arieli.
          Paraconsistent declarative semantics for extended logic
          programs. {\em Annals of Mathematics and Artificial Intelligence\/}~36(4),
          pages~381-417, 2002.

    \item O.Arieli, M.Denecker.
          Reducing preferential paraconsistent reasoning to classical
          entailment. {\em Journal of Logic and Computation\/}~13(4),
          pages~557-580, 2003.

    \item O.Arieli.
          Reasoning with different levels of uncertainty.
          {\em Journal of Applied Non-Classical Logics\/}~13(3--4),
          pages~317--343, 2003.

    \item O.Arieli, M.Denecker, B.Van Nuffelen, M.Bruynooghe.
          Coherent integration of databases by abductive logic programming.
          {\em Journal of Artificial Intelligence Research\/}~21,
          pages~245--286, 2004.

    \item O.Arieli, M.Denecker, B.Van Nuffelen, M.Bruynooghe.
          Computational methods for database repair by signed formulae.
          {\em Annals of Mathematics and Artificial Intelligence\/}~46(1--2),
          pages~4--37, 2006.

    \item C.Cornelis, O.Arieli, G.Deschrijver, E.Kerre.
          Uncertainty modeling by bilattice-based squares and triangles.
          {\em IEEE Transactions on Fuzzy Systems\/}~15(2), pages~161--175, 2007.

    \item G.Deschrijver, O.Arieli, C.Cornelis, E.Kerre.
          A bilattice-based framework for handling graded truth and imprecision.
          {\em Uncertainty, Fuzziness and Knowledge-Based Systems\/}~15(1),
          pages~13--41, 2007.

    \item O.Arieli.
          Paraconsistent reasoning and preferential entailments by signed
          quantified Boolean formulae. {\em ACM Transactions on Computational Logic\/}~8(3),
          Article~18, 2007.

    \item O.Arieli, M.Denecker, M.Bruynooghe.
          Distance semantics for database repair.
          {\em Annals of Mathematics and Artificial Intelligence\/}~50(3--4),
          pages~389--415, 2007.

    \item O.Arieli.
          Distance-based paraconsistent logics.
          {\em The International Journal of Approximate Reasoning\/}~48(3), special
          issue on non-monotonic reasoning and uncertainty, pages~766--783, 2008.

    \item O.Arieli.
          Reasoning with prioritized information by iterative aggregation of
          distance functions. {\em Journal of Applied Logic\/}~6(4), pages~589--605, 2008.

    \item O.Arieli, A.Zamansky.
          Distance-based non-deterministic semantics for reasoning with uncertainty.
          {\em The Logic Journal of the Interest Group in Pure and Applied Logics\/}~17(4),
          pages~325--350, 2009.

    \item M.Denecker, A.Cort\'es-Calabuig, M.Bruynooghe, O.Arieli.
          Towards a logical reconstruction of a theory for locally closed databases.
          {\em ACM Transactions on Database Systems\/}~35(3), Article~22, 2010.

    \item O.Arieli, A.Zamansky.
          Simplified forms of computerized reasoning with distance semantics.
          {\em Journal of Applied Logic\/}~9(1), pages 1--22, 2011.

    \item O.Arieli, A.Zamansky.
          A framework for reasoning under uncertainty based on non-deterministic
          distance semantics.
          {\em International Journal of Approximate Reasoning\/}~52(2), pages 184--211, 2011.

    \item O.Arieli, A.Avron, A.Zamansky.
          Maximal and pre-maximal paraconsistency in the framework of three-valued semantics.
          {\em Studia Logica\/}~97(1), pages~31--60, 2011.

    \item O.Arieli, A.Avron, A.Zamansky.
          Ideal paraconsistent logics.
          {\em Studia Logica\/}~99(1--3), pages~31--60, 2011.

    \item O.Arieli, M.Caminada.
          A QBF-based formalization of abstract argumentation semantics.
          {\em Journal of Applied Logic\/}~11(2), pages~229--252, 2013.

    \item O.Arieli, A.Zamansky.
          A dissimilarity-based framework for generating inconsistency-tolerant logics.
          {\em Annals of Mathematics and Artificial Intelligence\/}~73(1--2),
          pages~47--73, 2015.

    \item O.Arieli, C.Stra{\ss}er.
          Sequent-based logical argumentation.
          {\em Journal of Argument and Computation\/}~6(1), pages~73--99, 2015.

    \item O.Arieli.
          Conflict-free and conflict-tolerant semantics for constrained argumentation frameworks.
          {\em Journal of Applied Logic\/}~13(4), pages~582--604, 2015.

    \item O.Arieli, A.Avron.
          Three-valued paraconsistent propositional logics.
          In: {\em New Directions in Paraconsistent Logic\/}, Chapter~4,
          J.Y.B\'eziau, M.Chakraborty and S.Dutta, editors,
          pages~91--129, Springer, 2015.

    \item O.Arieli, A.Zamansky.
          A graded approach to database repair by context-aware distance semantics.
          {\em Fuzzy Sets and Systems\/}~298, pages~4--21, 2016.

    \item O.Arieli, C.Stra{\ss}er.
          Deductive argumentation by enhanced sequent calculi and dynamic derivations.
          {\em Electronic Notes in Theoretical Computer Science\/}~323, pages~21--37, 2016.

    \item O.Arieli.
          On the acceptance of loops in argumentation frameworks.
          {\em Journal of Logic and Computation\/}~26(4), pages~1203-1234, 2016.

    \item O.Arieli, A. Avron.
            Four-valued paradefnite logics. {\em Studia Logica\/}~105(6), pages~1087--1122, 2017.

   \item O.Arieli, A.Borg, C.Stra{\ss}er.
           Reasoning with maximal consistency by  argumentative approaches.
           {\em Journal of Logic and Computation\/}~28(7), pages~1523-–1563, 2018.

   \item C.Stra{\ss}er, O.Arieli.
            Normative reasoning by sequent-based argumentation.
            {\em Journal of Logic and Computation\/}~29(3), pages~387--415, 2019.

   \item O.Arieli, C.Stra{\ss}er.
             Logical argumentation by dynamic proof systems.
             {\em Theoretical Computer Science\/}~781, pages~63--91, 2019.

   \item O.Arieli, A.Borg, J.Heyninck.
            A review of the relations between logical argumentation and reasoning with maximal
            consistency. {\em Annals of Mathematics and Artificial Intelligence\/}~87(3),
            pages~187--226, 2019.

   \item J.Heyninck, O.Arieli.
           Simple contrapositive assumption-based argumentation frameworks.
           {\em Journal of Approximate Reasoning\/}~121, pages~103--124, 2020.

   \item A.Borg, C.Stra{\ss}er, O.Arieli.
           A generalized proof-theoretic approach to logical argumentation based on hypersequents.
           {\em Studia Logica\/}~109(1), pages 167--238, 2021.

  \item O.Arieli, A.Borg, J.Heyninck, C.Stra{\ss}er.
          Logic-based approaches to formal argumentation.
          {\em Journal of Applied Logics The IfCoLog Journal of Logics and their Applications\/}~8(6),
          pages 1793--1898, 2021.

  \item O.Arieli, J.Heyninck.
           Simple contrapositive assumption-based argumentation.\ Part~II: Reasoning with preferences.
           {\em Journal of Approximate Reasoning\/}~139, pages~28--53, 2021.

  \item O.Arieli, A.Borg, C.Stra{\ss}er. 
           A postulate-deriven study of logical argumentation  
           {\em Journal of Artificial Intelligence\/}~322, paper~103966, 2023.

\end{enumerate}


\section*{Refereed Papers in Conference Proceedings}

\begin{enumerate}

   \item O.Arieli, A.Avron.
         Logical bilattices and inconsistent data. {\em Proceedings of the 9th Annual
         IEEE Symposium on Logic in Computer Science\/} (LICS'94), pages~468--476,
         IEEE Press, July~1994.

   \item O.Arieli, A.Avron.
         A bilattice-based approach to recover consistent data from
         inconsistent knowledge-bases. {\em Proceedings of the 4th Bar-Ilan Symposium on
         Foundations of Artificial Intelligence\/} (BISFAI'95),
         M.Koppel and E.Shamir, editors, pages~14--23, AAAI Press, June 1995.

   \item O.Arieli, A.Avron.
         Automatic diagnoses for properly stratified knowledge-bases.
         {\em Proceedings of the 8th IEEE International Conference on Tools with
         Artificial Intelligence\/} (ICTAI'96), pages~392--399, IEEE Press,
         November 1996.

   \item O.Arieli, A.Avron.
         Four-valued diagnoses for stratified knowledge-bases.
         {\em Proceedings of the Annual Conference of the European Association for
         Computer Science Logic\/} (CSL'96), September 1996 -- Selected Papers.
         Lecture Notes in Computer Science No.1258, D.Van-Dalen and M.Benzem,
         editors, pages~1--17, Springer-Verlag, 1997.

   \item O.Arieli.
         A four-valued approach for handling inconsistency in prioritized
         knowledge-bases. {\em Proceedings of the 10th Annual Conference of
         the Florida Artificial Intelligence Research Society\/}
         (FLAIRS'97), pages~92--96, May~1997.

   \item O.Arieli, A.Avron.
         The logical role of the four-valued bilattice.
         {\em Proceedings of the 13th IEEE Annual Symposium on Logic in Computer
         Science\/} (LICS'98), pages~218--226, IEEE~Press, June~1998.

   \item O.Arieli.
         Four-valued logics for reasoning with uncertainty in prioritized
         data. {\em Proceedings of the 7th Conference on Information Processing
         and Management of Uncertainty in Knowledge-Base Systems\/}
         (IPMU'98), pages~503--510, Editions EDK publishers, July~1998.

   \item O.Arieli, A.Avron.
         Nonmonotonic and paraconsistent reasoning: From basic entailments
         to plausible relations. {\em Proceedings of the 5th European Conference on
         Symbolic and Quantitative Approaches to Reasoning with
         Uncertainty\/} (ESCSQARU'99), Lecture Notes in
         Artificial Intelligence No.1638, A.Hunter and S.Parsons, editors,
         pages~11--22, Springer-Verlag, July 1999.

   \item O.Arieli.
         An algorithmic approach to recover inconsistent knowledge-bases.
         {\em Proceedings of the 7th European Conference on Logics in Artificial
         Intelligence\/} (JELIA'00), Lecture Notes in Artificial Intelligence
         No.1919, M.Ojeda-Aciego, I.P.de~Guzman, G.Brewka, and L.M.Pereira,
         editors, pages~148--162, Springer, September 2000.

   \item O.Arieli.
         Reasoning with modularly pointwise preferential relations.
         {\em Proceedings of the 12th Belgian--Dutch Artificial Intelligence
         Conference\/} (BNAIC'00), A. van den Bosch and H. Weigand, editors,
         pages~61--68, BNVKI Association, November 2000.

   \item O.Arieli. B.Van Nuffelen, M.Denecker, M.Bruynooghe.
         Coherent composition of distributed knowledge-bases through
         abduction. {\em Proceedings of the 8th International Conference on
         Logic Programming, Artificial Intelligence and Reasoning\/}
         (LPAR'01), Lecture Notes in Computer Science No.2250,
         A.Nieuwenhuis, and A.Voronkov, editors, pages~620--635,
         Springer, December 2001.

   \item O.Arieli, M.Denecker.
         Modeling paraconsistent reasoning by classical logic.
         {\em Proceedings of the 2nd International Symposium on Foundations of
         Information and Knowledge Systems\/} (FoIKS'02),
         Lecture Notes in Computer Science No.2284, T.Eiter and
         K.D.Schewe, editors, pages~1--14, Springer, February 2002.

   \item O.Arieli.
         Paraconsistent semantics for extended logic programs.
         {\em Proceedings of the International Conference of Artificial Intelligence\/}
         (IC-AI'02), H.R.Arabnia and Y.Mun, editors,
         Vol.III, pages~1199--1205, CMSRA Press, June 2002.

   \item O.Arieli, M.Denecker, B.Van Nuffelen, M.Bruynooghe.
         Repairing inconsistent databases: A model-theoretic approach and
         abductive reasoning. {\em Proceedings of the ICLP'02 Workshop on
         Paraconsistent Computational Logic\/} (PCL'02),
         Federated Logic Conference (FLoC'02),
         H.Decker, J.Villadsen and T.Waradai, editors, Datalogiske Skrifter
         Vol.95, pages~51--65, Roskilde University, July 2002.

   \item O.Arieli.
         Preferential logics for reasoning with graded uncertainty.
         {\em Proceedings of the 7th European Conference on Symbolic and Quantitative
         Approaches to Reasoning with Uncertainty\/} (ECSQARU'03),
         Lecture Notes in Artificial Intelligence No.2711,
         T.D.Nielsen and N.L.Zhang, editors, pages~515--527,
         Springer, July 2003.

   \item O.Arieli, M.Denecker, B.Van Nuffelen, M.Bruynooghe.
         Database repair by signed formulae. {\em Proceedings of the 3rd
         International Symposium on Foundations of Information and
         Knowledge Systems\/} (FoIKS'04),
         Lecture Notes in Computer Science No.2942, D.Seipel and
         J.Turell--Torres, editors, pages~14--30, Springer, February 2004.

   \item B.Van Nuffelen, A.Cort\'es-Calabuig, M.Denecker, O.Arieli, M.Bruynooghe.
         Data integration using ID-logic. {\em Proceedings of the 16th
         International Conference on Advanced Information Systems
         Engineering\/} (CAiSE'04), Lecture Notes in
         Computer Science No.3084, A.Persson and J.Stirna, editors,
         pages~67--81, Springer, June 2004.

   \item O.Arieli.
         Paraconsistent preferential reasoning by signed quantified Boolean
         formulae. {\em Proceedings of the 16th European Conference on Artificial
         Intelligence\/} (ECAI'04), R.L\'opez de M\'antaras
         and L.Saitta, editors, pages~773--777, IOS Press, August 2004.

   \item O.Arieli, C.Cornelis, G.Deschrijver, E.Kerre.
         Relating intuitionistic fuzzy sets and interval-valued fuzzy sets
         through bilattices. {\em Proceedings of the 6th International FLINS Conference
         on Applied Computational Intelligence\/} (FLINS'04),
         D.Ruan et~al., editors, World Scientific, pages~57--64, September 2004.

   \item O.Arieli, C.Cornelis, G.Deschrijver, E.Kerre.
         Bilattice-based squares and triangles.  {\em Proceedings of the 8th European
         Conference on Symbolic and Quantitative Approaches to Reasoning with
         Uncertainty\/} (ECSQARU'05), Lecture Notes in Artificial Intelligence
         No.3571, L.Godo, editor, pages~563--575, Springer, July 2005.

   \item B.Van-Nuffelen, O.Arieli, A.Cort\'es-Calabuig, M.Bruynooghe.
         An ID-logic formalization of the composition of autonomous databases.
         {\em Proceedings of the 8th International Conference on Logic Programming and
         Nonmonotonic Reasoning\/} (LPNMR'05),
         Lecture Notes in Artificial Intelligence No.3662, C.Baral, G.Greco,
         N.Leone and G.Tarracina, editors, pages~132--144, Springer, September 2005.

   \item A.Cort\'es-Calabuig, M.Denecker, O.Arieli, B.Van-Nuffelen, M.Bruynooghe.
         On the local closed-world assumption of data-sources. {\em Proceedings of the 8th
         International Conference on Logic Programming and Nonmonotonic
         Reasoning\/} (LPNMR'05),
         Lecture Notes in Artificial Intelligence No.3662, C.Baral, G.Greco, N.Leone
         and G.Tarracina, editors, pages~145--157, Springer, September 2005.

   \item O.Arieli, C.Cornelis, G.Deschrijver.
         Preference modeling by rectangular bilattices. {\em Proceedings of the 3rd
         International Conference on Modeling Decisions for
         Artificial Intelligence\/} (MDAI'06),
         Lecture Notes in Artificial Intelligence No.3885, V.Torra et al.,
         editors, pages~22--33, Springer, April 2006.

   \item O.Arieli.
         Distance-based semantics for multiple-valued logics. {\em Proceedings of the 11th
         International Workshop on Non-Monotonic Reasoning \/} (NMR'06),
         J.Dix and A.Hunter, editors, pages~153--161, June 2006.

   \item O.Arieli, M.Denecker, M.Bruynooghe.
         Distance-based repairs of databases. {\em Proceedings of the 10th European Conference
         on Logics in Artificial Intelligence\/} (JELIA'06),
         Lecture Notes in Artificial Intelligence No.4160, M.Fisher et al.,
         editors, pages~43--55, Springer, September 2006.

   \item A.Cort\'es-Calabuig, M.Denecker, O.Arieli, M.Bruynooghe.
         Representation of partial knowledge and query answering in locally
         complete databases. {\em Proceedings of the 13th International Conference
         on Logic for Programming, Artificial Intelligence and Reasoning\/} (LPAR'06),
         Lecture Notes in Computer Science No.4246, M.Hermann and A.Voronkov, editors,
         pages~407--421, Springer, November 2006.

   \item O.Arieli.
         Paraconsistent reasoning and distance minimization. {\em Proceedings of the 3rd
         Conference on Computability in Europe\/} (CiE'07),
         In: Quaderni del Dipartimento di Scienze Matematiche e Informatiche
         'Roberto Magari', Universita  di~Siena, S.B.Cooper, B.Loewe, and A.Sorbi,
         editors, pages~53--61, June 2007.

   \item O.Arieli.
         Commonsense reasoning by distance semantics. {\em Proceedings of the 11th Conference
         on Theoretical Aspects of Rationality and Knowledge\/} (TARK'07),
         D.Samet, editor, pages~33--41, UCL Press, June, 2007.

   \item A.Cort\'es-Calabuig, M.Denecker, O.Arieli, M.Bruynooghe.
         Approximate query answering in locally closed databases.
         {\em Proceedings of the 22nd National Conference on Artificial Intelligence\/} (AAAI'07),
         AAAI~Press, pages~397--402, July 2007.

   \item O.Arieli.
         Reasoning with prioritized data by aggregation of distance functions.
         {\em Proceedings of the 1st Conference on Artificial General Intelligence\/} (AGI'08),
         Frontiers in Artificial Intelligence, Volume~171,
         P.Wang, B.Goertzel, and S.Franklin, editors, pages~27--38, IOS Press,
         March 2008.

   \item O.Arieli, A.Zamansky.
         Distance-based non-deterministic semantics.
         {\em Proceedings of the 1st Conference on Artificial General Intelligence\/} (AGI'08),
         Frontiers in Artificial Intelligence, Volume~171,
         P.Wang, B.Goertzel, and S.Franklin, editors, pages~39--50, IOS Press,
         March 2008.

   \item O.Arieli, A.Zamansky.
         Some simplified forms of reasoning with distance-based entailments.
         {\em Proceedings of the 21st Canadian Conference on Artificial Intelligence\/} (AI'08),
         Lecture Notes in Artificial Intelligence No.5032,
         S.Bergler, editor, pages~36--47, Springer, May 2008.

   \item A.Cort\'es-Calabuig, M.Denecker, O.Arieli, M.Bruynooghe.
         Efficient fixpoint methods for approximate query answering in locally complete
         databases. {\em Proceedings of LID'2008 -- Logic in Databases\/}, May 2008.

   \item O.Arieli, A.Zamansky.
         Reasoning with uncertainty by Nmatrix--metric semantics.
         {\em Proceedings of the 15th Workshop on Logic, Language, Information and
         Computation\/} (WoLLIC'08), Lecture Notes Artificial
         Intelligence No.5110, W.Hodges and R.de-Queiroz, editors, pages~69--82,
         Springer, July 2008.

   \item A.Cort\'es-Calabuig, M.Denecker, O.Arieli, M.Bruynooghe.
         Accuracy and efficiency of fixpoint methods for approximate query answering in
         locally complete databases. {\em Proceedings of the 11th International Conference
         on Principles of Knowledge Representation and Reasoning\/} (KR'08),
         G.Brewka and J.Lang, editors, pages~81--91, AAAI~Press, September 2008.

   \item O.Arieli, A.Zamansky.
         Non-deterministic distance semantics for handling incomplete and inconsistent data.
         {\em Proceedings of the 10th European Conference on Symbolic and Quantitative Approaches to
         Reasoning with Uncertainty\/} (ECSQARU'09), Lecture Notes in Artificial Intelligence
         No.5590, C.Sossai and G.Chemello, editors, pages~793--804, Springer, July 2009.

   \item O.Arieli.
         On the application of the Disjunctive Syllogism in paraconsistent logics based on
         four states of information. {\em Proceedings of the 12th International Conference
         on Principles of Knowledge Representation and Reasoning\/} (KR'10),
         pages~302--309, AAAI~Press, May 2010.

   \item O.Arieli, A.Avron, A.Zamansky.
         Maximally paraconsistent three-valued logics. {\em Proceedings of the 12th International
         Conference on Principles of Knowledge Representation and Reasoning\/} (KR'10),
         pages~310--318, AAAI~Press, May 2010.

   \item A.Avron, O.Arieli, A.Zamansky.
         On strong maximality of paraconsistent finite-valued logics.
         {\em Proceedings of the 25th Annual IEEE Symposium on Logic in Computer Science\/} (LICS'10),
         pages~304--313, IEEE~Press, July 2010.

   \item O.Arieli, A.Zamansky.
         Similarity-based inconsistency-tolerant logics.
         {\em Proceedings of the 12th European Conference on Logics in Artificial Intelligence\/}
         (JELIA'10), Lecture Notes in Artificial Intelligence No.6341,
         T.Janhunen and I.Niemel{\"a}, editors, pages~11--23, Springer, September 2010.

   \item O.Arieli, A.Avron, A.Zamansky.
         What is an ideal logic for reasoning with inconsistency?
         {\em Proceedings of the 22nd International Joint Conference on Artificial Intelligence\/}
         (IJCAI'11), T.Walsh, editor, pages~706-711, AAAI~Press, July 2011.

   \item O.Arieli, A.Zamansky.
         Inconsistency-tolerance in knowledge-based systems by dissimilarities.
         {\em Proceedings of the 7th International Symposium on Foundations of Information and Knowledge Systems\/}
         (FoIKS'12), Lecture Notes in Computer Science No.7153,
         Th.Lukasiewicz and A.Sali, editors, pages~34--50, Springer, March 2012.

   \item O.Arieli, M.W.A.Caminada.
         A general QBF-based formalization of abstract argumentation theory.
         {\em Proceedings of the 4th International Conference on Computational Models of Argument\/}
         (COMMA'12), Frontiers in Artificial Intelligence and Applications Vol.245,
         B. Verheij, S. Szeider and S. Woltran editors, pages~105--116, IOS~Press, September 2012.

   \item O.Arieli.
         Conflict-tolerant semantics for argumentation frameworks.
         {\em Proceedings of the 13th European Conference on Logics in Artificial Intelligence\/}
         (JELIA'12), Lecture Notes in Artificial Intelligence No.7519,
         L.Fari{\~n}as del Cerro, A.Herzig and J.Mengin editors, pages~28--40,
         Springer, September 2012.

   \item O.Arieli.
         Towards constraints handling by conflict tolerance in abstract argumentation frameworks.
         {\em Proceedings of the 26th International Florida Artificial Intelligence Research Society Conference
         (FLAIRS'13), Special Track on Uncertain Reasoning\/}, Ch.Boonthem-Denecke and G.M.Youngblood,
         editors, pages~585--560, AAAI~Press, May 2013.

   \item O.Arieli.
         A sequent-based representation of logical argumentation.
         {\em Proceedings of the 14th International Workshop on Computational Logic in Multi-Agent Systems (CLIMA'13),
         Special Session on Argumentation Technologies\/}, J.Leite, T.C.Son, P.Torroni, L.van der Torre
         and S. Woltran, editors, Lecture Notes in Artificial Intelligence No.8143, pages~69--85,
         Springer, September 2013.

   \item A.Zamansky, O.Arieli, K.Stefanidis.
         Context-aware distance semantics for inconsistent database systems.
         {\em Proceedings of the 15th International Conference on Information Processing and Management of Uncertainty
         in Knowledge-Base Systems\/} (IPMU'14) Part~II, Communications in Computer and Information Sciences
         Volume~443, A.Laurent~et~al., editors, pages~194--203, Springer, July 2014.

   \item C.Stra{\ss}er, O.Arieli.
         Sequent-based argumentation for normative reasoning.
         {\em Proceedings of the 12th International Conference on Deontic Logic and Normative Systems\/} (DEON'2014),
         Lecture Notes in Artificial Intelligence No.8554, F.Cariani~et~at., editors, pages~224--240,
         Springer, July 2014.

   \item O.Arieli, T.Reinstra.
         Preferential reasoning based on abstract argumentation semantics.
         {\em Proceedings of the 5th International Conference on Computational Models of Argument\/} (COMMA'14),
         Frontiers in Artificial Intelligence and Applications Volume~266, S.Parsons~et~at., editors,
         pages~77--88, IOS~Press, September 2014.

   \item O.Arieli, C.Stra{\ss}er.
         Dynamic derivations for sequent-based logical argumentation.
         {\em Proceedings of the 5th International Conference on Computational Models of Argument\/} (COMMA'14),
         Frontiers in Artificial Intelligence and Applications Vol.266, S.Parsons~et~at., editors,
         pages~89--100, IOS~Press, September 2014.

   \item O.Arieli, C.Stra{\ss}er.
         Argumentative approaches to reasoning with maximal consistency.
         {\em Proceedings of the 15th International Conference on Principles of Knowledge Representation
         and Reasoning\/} (KR'16), C.Baral, J.Delgrande and F.Wolter, editors, pages~509--512,
         AAAI~Press, April 2016.

   \item O.Arieli, A.Avron.
         Minimal paradefinite logics for reasoning with incompleteness and inconsistency.
         {\em Proceedings of the 1st International Conference on Formal Structures for Computation and Deduction\/}
         (FSCD'16), D.Kesner and B.Pientka, editors, LIPIcs Volume~52, pages~7:1--7:15, June 2016.

   \item O.Arieli, A.Borg, C.Stra{\ss}er.
         Argumentative approaches to reasoning with consistent subsets of premises.
         {\em Proceedings of the 30th International Conference on Industrial, Engineering, Other Applications
         of Applied Intelligent Systems\/} (IEA/AIE'17), Lecture Notes in Artificial Intelligence No.10350,
         S.Benferhat, K.Tabia, and M.Ali., editors, pages~455--465, Springer, June 2017.

  \item A.Borg, O.Arieli, C.Stra{\ss}er.
         Hypersequent-based argumentation: An instantiation in the relevance logic {RM}.
         {\em  Proceedings of the 4th International Workshop on Theory and Applications of Formal Argument\/}
         (TAFA'17), August 2017. Post proceedings (revised selected papers) In: Lecture Notes
         in Artificial Intelligence No.10757, E.Black, S.Modgil, and N.Oren, editors, pages~17--34,
         Springer, 2018.

  \item J.Heyninck, O.Arieli.
          On the semantics of simple contrapositive assumption-based argumentation frameworks.
          {\em Proceedings of the 2nd Chinese Conference on Logic and Argumentation\/} (CLAR'18), June 2018.

  \item A.Borg, O.Arieli.
          Hypersequential argumentation frameworks: An instantiation in the modal logic {S5}.
          {\em  Proceedings of the 17th International Conference on Autonomous Agents and Multiagent Systems\/}
          (AAMAS'18), M.Dastani, G.Sukthankar, E.Andr\'e, and S.Koenig, editors, pages~1097--1104,
          ACM Press, July 2018.

  \item O.Arieli, A.Borg, C.Stra{\ss}er.
          Prioritized sequent-based argumentation. {\em  Proceedings of the 17th International Conference on
          Autonomous Agents and Multiagent Systems\/} (AAMAS'18), M.Dastani, G.Sukthankar, E.Andr\'e,
          and S.Koenig, editors, pages~1105--1113, ACM~Press, July 2018.

  \item J.Heyninck, O.Arieli.
          On the semantics of simple contrapositive assumption-based argumentation frameworks.
          {\em Proceedings of the 7th International Conference on Computational Models of Argument\/} (COMMA'18),
          Frontiers in Artificial Intelligence and Applications Vol.305, S.Modgil, K.Budzynska, and J.Lawrence, editors,
          pages~9--20, IOS~Press, September 2018.

  \item J.Heyninck, O.Arieli.
          Simple contrapositive assumption-based frameworks - Extended abstract.
          {\em  Proceedings of the 18th International Conference on  Autonomous Agents and Multiagent Systems\/}
          (AAMAS'19), N.Agmon, M.Taylor, E.Elkind and M.Veloso, editors, pages~2018--2020, ACM~Press, May 2019.

  \item J.Heyninck, O.Arieli.
          Simple contrapositive assumption-based frameworks.
          {\em  Proceedings of the 15th International Conference on Logic Programming and Nonmonotonic Reasoning\/}
          (LPNMR'19), Lecture Notes in Computer Science No.11481, M.Balduccini, Y.Lierler and S.Woltran, editors,
          pages~75--88, Springer, June~2019.

  \item J.Heyninck, O.Arieli.
          An argumentative characterization of disjunctive logic programming.
          {\em  Proceedings of the 19th EPIA Conference on Artificial Intelligence\/} (EPIA'19),
          Lecture Notes in Artificial Intelligence No.11805, P. Moura~Oliveira et al. editors, pages~526--538,
          Springer, September 2019.

  \item O.Arieli, A.Borg, C.Stra{\ss}er.
           Tuning logical argumentation frameworks: A postulate-derived approach.
           {\em Proceedings of the 33rd International Florida Artificial Intelligence Research Society Conference\/}
           (FLAIRS-33), Special Track on Uncertain Reasoning, pages~557--562, AAAI~Press, May 2020.

  \item O.Arieli, J.Heyninck.
          Prioritized simple contrapositive assumption-based frameworks.
          {\em Proceedings of the 24th European Conference on Artificial Intelligence\/} (ECAI'20),
           Frontiers in Artificial Intelligence and Applications Vol.325, G.De~Giacomo et al., editors,
           pages~608--615. IOS~Press, August 2020.

  \item O.Arieli, C.Stra{\ss}er.
           On minimality and consistency tolerance in logical argumentation frameworks.
           {\em Proceedings of the 8th International Conference on Computational Models of Argument\/} (COMMA'20),
           Frontiers in Artificial Intelligence and Applications Vol.326, IOS press, H.Prakken et al., editors,
           pages~91--102, IOS Press, September 2020.

  \item J.Heyninck, O.Arieli.
           Argumentative reflections of approximation fixpoint theory.
           {\em Proceedings of the 8th International Conference on Computational Models of Argument\/} (COMMA'20),
           Frontiers in Artificial Intelligence and Applications Vol.326, IOS press, H.Prakken et al., editors,
           pages~215--226, IOS~Press, September 2020.

  \item O.Arieli, A.Borg, C.Stra{\ss}er.
          Characterizations and classifications of argumentative entailments.
          {\em Proceedings of the 18th International Conference on Knowledge Representation and Reasoning\/} (KR'21),
          pages~52--62, M.Bienvenu, G.Lakemeyer and E.Erdem, editors, IJCAI Organization, November 2021.

  \item J.Heyninck, O.Arieli.
          Approximation fixpoint theory for non-deterministic operators and its application in disjunctive logic programming.
          {\em Proceedings of the 18th International Conference on Knowledge Representation and Reasoning\/} (KR'21),
          pages~334--344, M.Bienvenu, G.Lakemeyer and E.Erdem, editors, IJCAI Organization, November 2021.

  \item O.Arieli, K.van~Berkel, C.Stra{\ss}er.
         Annotated sequent calculi for paraconsistent reasoning and their relations to logical argumentation.
         {\em Proceedings of the 31st International Joint Conference on Artificial Intelligence\/} (IJCAI'22),
         L.De~Raedt, editor, pages 2532--2538, ijcai.org, July 2022.

  \item O.Arieli, A.Borg, M.Hesse, C.Stra{\ss}er.
         Abductive reasoning with sequent-based argumentation (Extended abstract).
         {\em Proceedings of the 20th International Workshop on Non-Monotonic Reasoning\/} (NMR'22),
         CEUR Workshop Proceedings No.3197, O.Arieli, G.Casini and L.Giordano, editors, pages~143--146, CEUR-WS.org, August 2022.

  \item O.Arieli, A.Borg, M.Hesse, C.Stra{\ss}er.
           Explainable logic-based argumentation.
           {\em Proceedings of the 9th International Conference on Computational Models of Argument\/} (COMMA'20),
           Frontiers in Artificial Intelligence and Applications Vol.353, F.Toni et al., editors, pages~32--43, IOS press, September 2022.

  \item O.Arieli, J.Heyninck.
           Simple contrapositive assumption-based argumentation with partially-ordered preferences
           {\em Proceedings of the 20th International Conference on Knowledge Representation and Reasoning\/} (KR'23),
           Accepted, 2023.

\end{enumerate}


\section*{Book Chapters}

\begin{enumerate}

 \item O. Arieli, C. Cornelis, G. Deschrijver, E. E. Kerre.        
          Relating intuitionistic fuzzy sets and interval-valued fuzzy sets through bilattices. In: {\em Applied Computational Intelligence\/}.
          D. Ruan, P. D'hondt, M. De Cock, M. Nachtegael, and E.E.Kerre, editors, World Scientific, pages~57--64, 2004.

  \item O.Arieli, A.Zamansky.
          Introduction: Non-classical Logics -- Between Semantics and Proof Theory (In Relation to Arnon Avron's Work)
          In: {\em Arnon Avron on Semantics and Proof Theory of Non-Classical Logics\/}, Volume~21 of Outstanding
          Contributions to Logic, pages 1--11, O.Arieli and A.Zamansky, editors, Springer, 2021.

  \item O.Arieli, A.Borg, J.Heyninck, C.Stra{\ss}er.
           Logic-Based Approaches to Formal Argumentation. In: Chapter~12 of the {\em Handbook of Formal Argumentation\/},
           Volume~II, D.Gabbay, M.Giacomin, G.Simari, M.Thimm, editors, College Publications, 2021.
           
  \item O.Arieli.
           Four-valued semantics for abstract argumentation frameworks using (extensions of) Dunn--{\allowbreak}
           Belnap four-valued logic. In: {\em Relevance Logics and other Tools for Reasoning: Essays in Honor of
           J.~Michael Dunn\/}, K.Bimb\'o, editor, Volume~46 of Tributes, pages~31--53, College Publications, 2022.         

\end{enumerate}


\section*{Editorship, Books}

\begin{enumerate}

   \item O.Arieli, A.Zamansk (eds). {\em Arnon Avron on Semantics and Proof Theory of Non-Classical Logics\/}.
           Outstanding Contributions to Logic, volume~21, Springer, 2021 (ISBN: 978-3-030-71258-7).

\end{enumerate}


\section*{Editorship, Journal Special Issues}

\begin{enumerate}

   \item O.Arieli, B.Konikowska, A.Rabinovich, A.Zamansky (eds).
         {\em Journal of Logic and Computation, Special Issue for IsraLog'11, on: Logic -- Between Semantics
         and Proof Theory (in honor of A.Avron's  60th Birthday)\/}, Volume~26, Issue~1,
         February~2016.

   \item O.Arieli, A.Zamansky (eds).
         {\em The Logic Journal of the Interest Group in Pure and Applied Logics
         {\rm (IGPL)}, Special Issue for IsraLog'14\/}, Volume~24, Number~3, June~2016.

   \item O.Arieli, A.Zamansky (eds).
         {\em Journal of Applied Logic, Special Issue for IsraLog'17\/}, Volume~6, Number~2, March~2019.

\end{enumerate}


\section*{Editorship, Conference Proceedings}

\begin{enumerate}

   \item O.Arieli, G.Casini, L.Giordano (eds).
           {\em Proceedings of the 20th International Workshop on Non-Monotonic Reasoning\/} (NMR'2022),
            CEUR Workshop Proceedings, Volume 3197, CEUR-WS.org, August 2022.

   \item O.Arieli, M.Homola, J.C.Jung, M.Mugnier: (eds).
           {\em Proceedings of the 35th International Workshop on Description Logics\/} (DL'2022),
            CEUR Workshop Proceedings, Volume 3263, CEUR-WS.org, August 2022.

\end{enumerate}


\section*{Abstracts or Unpublished Presentations in Scientific Meetings}

\begin{enumerate}

    \item O.Arieli, A.Avron.
          Bilattices and paraconsistency. {\em First World Congress on
          Paraconsistency\/} (WCP'97), Ghent, Belgium, August 1997.

    \item O.Arieli, A.Avron.
          Using four values for computerized reasoning. {\em The 1998
          Annual Conference of the European Association for Computer
          Science Logic\/} (CSL'98). Brno, Czech Republic, August 1998.

    \item O.Arieli.
          Four-valued logics for reasoning with uncertainty. {\em The
          Benelux Workshop on Computational Logic\/} (BenCl'2000), Peer,
          Belgium, May 2000 (Abstract in Technical Report No.~CW290,
          H.Vandecasteele and M.Bruynooghe, editors, Department of
          Computer Science, University of Leuven, May 2000).

    \item O.Arieli.
          Database integration: preferential semantics and abductive
          reasoning. {\em Dagstuhl Seminar No.03241 on Inconsistency
          Tolerance in Database, Knowledge-bases, and Software Systems\/},
          Schloss Dagstuhl, Wadern, Germany, June 2003.
          {\tt ftp://ftp.dagstuhl.de/pub/Proceedings/03/03141/}.

   \item A.Cort\'es-Calabuig, M.Denecker, O.Arieli, B.Van-Nuffelen, M.Bruynooghe.
         On the local closed-world assumption of data-sources.
         {\em Biennial Israeli Symposium on the Foundations of Artificial
         Intelligence\/} (BISFAI'05), Haifa, Israel, June 2005.

   \item B.Van-Nuffelen, M.Denecker, O.Arieli, A.Cort\'es-Calabuig, M.Bruynooghe.
         An ID-logic formalization of the composition of autonomous databases.
         {\em Biennial Israeli Symposium on the Foundations of Artificial
         Intelligence\/} (BISFAI'05), Haifa, Israel, June 2005.

   \item A.Cort\'es-Calabuig, M.Denecker, O.Arieli, B.Van-Nuffelen, M.Bruynooghe.
         On the local closed-world assumption of data-sources.
         {\em Proceedings of the 17th Belgian-Dutch Conference on Artificial Intelligence\/}
         (BNAIC'05), Brussels, Belgium, BNVKI Association, pages 333--334,
         October 2005.

   \item A.Cort\'es-Calabuig, M.Denecker, O.Arieli, M.Bruynooghe.
         Representation of partial knowledge and reasoning in locally complete
         databases. {\em The Second International Workshop on Exchange and
         Integration of Data\/} (EID'06), Brixen-Bressanone, Italy, June 2006.

   \item A.Cortes-Calabuig, M.Denecker, O.Arieli, M.Bruynooghe. Using deductive
         databases technology for approximate query answering in partially
         complete databases. {\em Dutch-Belgian DataBase Day\/} (DBDBD-07),
         Eindhoven, The Netherlands, November 2007.

   \item A.Cort\'es-Calabuig, M.Denecker, O.Arieli, M.Bruynooghe.
         Approximate query answering in locally closed databases.
         {\em Proceedings of the 19th Belgian-Dutch Conference on Artificial Intelligence\/}
         (BNAIC'07), Utrecht, The Netherlands, pages 339--340, November 2007.

   \item O.Arieli, A.Zamansky.
         Non-deterministic distance semantics for handling incomplete and inconsistent data.
         {\em 10th Biennial Israeli Symposium on the Foundations of Artificial
         Intelligence\/} (BISFAI'09), Ramat-Gan, Israel, June 2009.

   \item O.Arieli, A.Avron, A.Zamansky.
         All natural three-valued paraconsistent logics are maximal.
         {\em The 3rd World Congress on Universal Logic\/} (UniLog'2010),
         {\em Special Session on Negation\/}. Lisbon, Portugal, April 2010.

   \item O.Arieli.
         A tutorial on bilattices.
         {\em Research Workshop on Duality Theory in Algebra, Logic and Computer Science\/}
         (Invited Tutorial), Oxford, UK, June 2012.

   \item O.Arieli, A.Zamansky.
         A dissimilarity-based approach to handling inconsistency in non-truth-functional logics.
         {\em The 4thd World Congress on Universal Logic\/} (UniLog'2013),
         {\em Workshop on Compositional Meaning in Logic\/}. Rio, Brazil, March 2013.

   \item A.Zamansky, O.Arieli, K. Stefanidis.
         Inconsistency management based on relevance degrees.
         {\em 35th Linz Seminar on Fuzzy Set Theory\/}. Linz, Austria, February 2014.
         (Abstract in: Proceedings of the 35th Linz Seminar on Fuzzy Set Theory, T.Flaminio, L.Godo, S.Gottwald
         and E.Klement, editors, pages 136--137, Johannes Kepler Universitat, A-4040 Linz, February 2014).

   \item O.Arieli, K. Stefanidis, A.Zamansky.
         Combining non-determinism and context awareness in consistency restoration systems.
         {\em The 2nd Workshop on Compositional Meaning in Logic\/} (GeTFun 2.0)
         Vienna Summer of Logic, Vienna, Austria, July 2014.

   \item O.Arieli, C.Stra{\ss}er.
         Sequent-based logical argumentation.
         {\em The 2nd Israeli  Workshop on Non-Classical Logics and Their Applications\/}
         (IsraLog'14), Haifa, Israel, September 2014.

   \item O.Arieli.
         Desirable properties of paraconsistent logics.
         {\em Dagstuhl Seminar 15211: Multi-disciplinary approaches to reasoning with imperfect
         information and knowledge -- a synthesis and a roadmap of challenges.\/} Dagstuhl Castle,
         Wadern, Germany, May 2015.

   \item O.Arieli, C.Stra{\ss}er.
         Sequent-based logical argumentation.
         {\em The 10th Workshop on Logical and Semantic Frameworks with Applications\/} (LFSA'15),
         Natal, Brazil, September 2015. (Invited Talk)

   \item O.Arieli, A.Avron.
         Paradefinite logics for reasoning with incompleteness and inconsistency.
         {\em The 3rd Workshop on Compositional Meaning in Logic\/} (GeTFun~3.0).
         Natal, Brazil, September 2015.

   \item O.Arieli, A.Avron.
         Paradefinite logics for reasoning with incompleteness and inconsistency.
         {\em The 4th Workshop on Compositional Meaning in Logic\/} (GeTFun~4.0).
         Coimbra, Portugal, July 2016.

   \item O.Arieli, A.Avron.
         Four-valued paradefinite logics.
         {\em The 2nd Prague Seminar on Paraconsistent Logics\/}.
         Prague, The Czech Republic, June 2017.

  \item A.Borg, O.Arieli, C.Stra{\ss}er.
           Hypersequent-based argumentation: An instantiation in the relevance logic RM.
          {\em The 3rd Israeli  Workshop on Non-Classical Logics and Their Applications\/}
          (IsraLog'17), Haifa, Israel, October 2017.

  \item O.Arieli, A.Bong, C.Stra{\ss}er.
          Reasoning with maximal consistency by argumentative approaches.
          {\em The 3rd  Madeira Workshop on Belief Revision, Argumentation, Ontologies, and Norms\/}
          (BRAON'17), Madeira, Portugal, November 2017.

  \item A.Borg, O.Arieli.
           Hypersequent-based argumentation: An instantiation in the modal logic S5.
          {\em The 6th World Congress on Universal Logics\/} (UniLog'18), Vichy, France, June 2018.

  \item O.Arieli. Structured argumentation frameworks and reasoning with maximal consistency.
          {\em Workshop on Bridging the Gap between Formal Argumentation and Actual Human Reasoning\/}
          (invited keynote) Bochum, Germany, October 2018.

  \item O.Arieli, A.Bong, C.Stra{\ss}er.
          Properties of argumentation frameworks: A proof-theoretical study.
          {\em The 3rd  European Conference on Argumentation\/} (ECA'19), Groningen, The Netherlands,
          June 2019.

  \item O.Arieli. Sequent-based argumentation frameworks: Some recent results. {\em Workshop on Logic \& Argumentation\/},
          Technical University Vienna, November 2019.

  \item O.Arieli, A.Bong, C.Stra{\ss}er.
          Logic-Based Approaches to Formal Argumentation: Chapter presentation of the second volume of HoFA.
          {\em Current Trends in Formal Argumentation\/} Bertinoro, Italy, November 2019.

   \item O.Arieli. Sequent-based deductive argumentation. Research seminar of the logic group, Department of Philosophy,
           University of Milan, Online, April 2020.

   \item O.Arieli. What is an argument? Consistency, minimality, and meta-linguistic considerations. {\em The 3rd workshop on
             argument strength\/}, Keynote talk, Online, October 2021.

   \item O.Arieli, A.Bong, C.Stra{\ss}er. Proof-theoretic approaches to logical argumentation. {\em The 18th International Conference
            on Principles of Knowledge Representation and Reasoning (KR’2021)\/}, Tutorial, Online, November 2021.
            
   \item O.Arieli. Argumentation-based approaches to paraconsistency. {\em São Paulo School of Advanced Science on Contemporary Logic, 
            Rationality and Information (SPLogiC'2023)\/}, Course, University of Campinas, Campinas, Brazil, February 2023. 

\end{enumerate}


\section*{Thesis}

\begin{enumerate}

%  \item O.Arieli.
%      {\em Methods of reasoning with inconsistency through bilattice.\/}
%      M.Sc. thesis (in Hebrew). Department of Computer Science, Tel-Aviv University, July 1992.

   \item O.Arieli.
         {\em Multiple-valued logics for reasoning with uncertainty.\/}
         Ph.D. Dissertation. Department of Computer Science, Tel-Aviv University, April 1999.

\end{enumerate}


\end{document}
